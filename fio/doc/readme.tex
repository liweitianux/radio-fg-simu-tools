%%%%%%%%%%%%%%%%%%%%%%%%%%%%%%%%%%%%%%%%%%%%%%%%%%%%%%%%%%%%%%%%%%%%%%%%%%%
%% FITSIO C++ 接口库 libfio 使用说明                                     %%
%%                                                                       %%
%% 顾俊骅                                                                %%
%% v1.0, June 7, 2007                                                    %%
%%                                                                       %%
%% 李维天                                                                %%
%% v1.1, May 8, 2012                                                     %%
%% minor changes                                                         %%
%%                                                                       %%
%% use `xelatex' to process this manual                                  %%
%% besides, maybe some extra packages and fonts needed                   %%
%%%%%%%%%%%%%%%%%%%%%%%%%%%%%%%%%%%%%%%%%%%%%%%%%%%%%%%%%%%%%%%%%%%%%%%%%%%

\documentclass[a4paper,12pt]{article}

\usepackage[margin=2.5cm]{geometry}
\usepackage{amsmath}
\usepackage[xetex,x11names]{xcolor}
\usepackage{graphicx}
\usepackage{fontspec}
\usepackage{unicode-math}
\usepackage{xltxtra}
\usepackage{xunicode}
\usepackage{xeCJK}
\usepackage{listings}
\usepackage{paralist}     % compactenum, compactitem, ...
% \usepackage{booktabs}     % improve the 'tabular' env
\usepackage{hyperref}
%\usepackage[all]{hypcap}  % help to make the link point to the top of
                          % the image. call this package *after* hyperref

\XeTeXinputencoding "UTF8"
\XeTeXlinebreaklocale "zh"
\XeTeXlinebreakskip = 0pt plus 1pt minus 0.1pt

\defaultfontfeatures{Mapping=tex-text}   % should before \set*font

%\setCJKfamilyfont{}{}
\setCJKmainfont[BoldFont={WenQuanYi Zen Hei},%
  ItalicFont={WenQuanYi Zen Hei},%
  SlantedFont={AR PL UKai CN}%
  ]{AR PLBaosong2GBK Light}
\setCJKsansfont{WenQuanYi Zen Hei}
\setCJKmonofont{WenQuanYi Zen Hei Mono}
\punctstyle{kaiming}
\setlength{\parindent}{2em}
\renewcommand{\baselinestretch}{1.25}

\setmainfont[%BoldFont={Linux Libertine O},%
  SmallCapsFont={Linux Libertine Capitals O},%
  SlantedFont={Linux Libertine Slanted O},%
  ]{Linux Libertine O}
\setsansfont[%BoldFont={Linux Biolinum O},%
  SmallCapsFont={Linux Biolinum Capitals O},%
  SlantedFont={Linux Biolinum Slanted O},%
  ]{Linux Biolinum O}
% \setmonofont[%BoldFont={CMU Typewriter Text},%
%   ]{CMU Typewriter Text}
\setmonofont{DejaVu Sans Mono}
\setmathfont{XITS Math}      % require 'unicode-math' pkg

\hypersetup{unicode=true,%
            pdfstartview={FitH},%                  % fits the width
            pdftitle={FITSIO C++ 接口库 libfio 使用说明},%
            pdfauthor={LIweitiaNux},%
            pdfkeywords={FITSIO, libfio},%
            pdfnewwindow=true,%                    % links in new window
            colorlinks=true,%                      % false: boxed links
            linkcolor=red,%
            citecolor=green,%
            filecolor=magenta,%
            urlcolor=cyan%
}


\lstset{basicstyle={\scriptsize,\ttfamily},%
        numbers=left,%
        numberstyle=\tiny,%
        keywordstyle=\bfseries\color{purple},%
        commentstyle=\itshape\color{blue},%
        stringstyle=\color{red},%
        showstringspaces=false,%
        frame=shadowbox,%
        breaklines=true,%
        escapeinside=``
}


\title{\textsc{FITSIO C++ 接口库 libf{}io 使用说明}}
\author{顾俊骅}
\date{v1.0, June 7, 2007 \\
  v1.1, May 8, 2012, Minor changes }

\begin{document}
\maketitle

\section{功能描述}
\texttt{libfio} 是 \texttt{cfitsio} 软件包在 \texttt{C++} 语言
上的包装。通过 \texttt{libfio},可以在 \texttt{C++} 语言中方便
地操纵 \texttt{fits} 格式的文件,主要是 \texttt{fits} 图像文件。
\texttt{libfio} 隐藏了 \texttt{cfitsio} 库中所要求的 \texttt{C}
语言指针语法,可以大大地提高效率,精简程序长度。


\section{安装}
\texttt{libfio} 依赖于 \texttt{cfitsio} 和 \texttt{blitz++} 两个库。

\subsection{安装 \texttt{blitz++}}
Debian Linux 官方源中有 \texttt{blitz++},分为 \texttt{libblitz0-dev},
\texttt{libblitz0ldbl} 和 \texttt{libblitz-doc} 三个包,可以使用
\texttt{apt} 或 \texttt{aptitude} 直接安装:
\begin{verbatim}
# apt-get install libblitz0-dev libblitz0ldbl libblitz-doc
\end{verbatim}

Ubuntu Linux 官方源中未包含 \texttt{blitz++} 软件包,但可以直接从
\texttt{launchpad.net} ( \url{https://launchpad.net/ubuntu/+source/blitz\%2B\%2B} )
下载编译好的二进制包,注意根据安装的系统选择 \texttt{i386} (32 位) 或者
\texttt{amd64} (64 位),然后使用 \texttt{dpkg} 安装,如:
\begin{verbatim}
# dpkg -i libblitz0-dev_0.9-12build1_amd64.deb \
    libblitz0ldbl_0.9-12build1_amd64.deb \
    libblitz-doc_0.9-12build1_all.deb
\end{verbatim}

对于其他 Linux 发行版,如果在源里找不到,也没有预编译好的二进制包,则
可手动编译安装。注意应该安装好相应的开发工具和依赖软件包。

\subsection{设置 \texttt{fitsio}}
更新:本软件包所需要的 \texttt{fitsio} 库和头文件现已整理存放于子目录
\texttt{cfitsio},所以可以直接进行之后的编译安装。

此处整理出来的 fitsio 等文件是从 \texttt{HEASOFT 6.12} 提取。


\section{编译安装 \texttt{libfio}}
解压本软件包后,进入解压后的目录,首先以``普通用户''执行以下命令编译:
\begin{verbatim}
$ make clean
$ make
\end{verbatim}
如果编译顺利通过,则可以切换到``超级用户''安装:
\begin{verbatim}
# make install
\end{verbatim}
默认安装路径为 \texttt{/usr/local/include} 和 \texttt{/usr/local/lib}。
同时在系统目录 \texttt{/usr/lib} 建立符号链接,否则编译后的程序可能因
无法找到所需的库而无法正常运行。
如果不满意,当然可以亲自修改 \texttt{Makefile}。


\section{使用 \texttt{libfio}}

\subsection{头文件包含}
首先应该在使用到 \texttt{libfio} 的程序源文件中包含对应的头文件
\texttt{fio.h}:
\begin{verbatim}
#include <fio.h>
\end{verbatim}
通常情况下,可能还需要再添加一行:
\begin{verbatim}
using namespace blitz;
\end{verbatim}

\subsection{编译选项}
如果将 \texttt{libfio} 库和相关的头文件安装于默认路径,即
\texttt{/usr/local/lib} 和 \texttt{/usr/local/include},则
在编译使用到 \texttt{libfio} 的程序源文件时,应该添加如下
编译选项:
\begin{verbatim}
-I/usr/local/include -L/usr/local/lib -lfio -lcfitsio -lblitz
\end{verbatim}
此外,建议添加选项 \texttt{-Wall} 来让编译器报告所有警告,还可以
添加选项 \texttt{-g} 以用于之后可在 \texttt{GDB} 中调试。如果
确认程序没有错误或不足,则可以使用选项 \texttt{-O2} 来打开编译
器的一些优化功能。

\subsection{程序实例}
下面通过分析一个实际程序,介绍 \texttt{libfio} 库的使用方法。
\begin{lstlisting}[language={C++}]
// `包含` fio `头文件`
#include <fio.h>
#include <iostream>
#include <cmath>

// `命名空间`
using namespace std;
using namespace blitz;

// `主程序开始`
int main(int argc, char *argv[])
// argc `和` argv `用于从命令行获取参数`
// argc `是命令行参数的数目,包含程序名称`
// argv[0], argv[1], ..., argv[argc-1] `是各个参数字符串`
{
    if (3 != argc) {        // `判断输入参数个数是否符合要求`
        fprintf(stderr, "Usage:\n");
        fprintf(stderr, "     %s in.fits out.fits\n", argv[0]);
        exit(-1);           // `强制程序结束并返回错误代码` -1
    }

    cfitsfile ff1;          // `声明一个用于操作` fits `文件的对象`
    ff1.open(argv[1]);      // `打开一个已经存在的` fits `文件`
    Array<double,2> img1;   // `声明一个用于存放图像的矩阵`
    ff1 >> img1;            // `将数据从` fits `文件导入矩阵`

    // `然后可以对这个矩阵中的数据进行各种处理`
    // `比如说,这里计算输入图像的泊松误差`
    Array<double,2> img2(img1.shape());
    int i, j;
    for (i=0; i<img2.extent(0); ++i) {
        for (j=0; j<img2.extent(1); ++j) {
            img2(i,j) = sqrt(img1(i,j));
        }
    }

    cfitsfile ff2;          // `声明另一个用于操作` fits `文件的对象`
    ff2.create(argv[2]);    // `新建一个` fits `文件`
                            // `如果文件存在,则先删除再新建`
    ff2 << img2;            // `将数据从矩阵导出到` fits `文件`

    return 0;               // `程序正常结束`
}

\end{lstlisting}


\section{ChangeLogs}
\noindent 2012/05/08, v1.1
\begin{compactitem}
  \item 更正 \texttt{fio.cc} 中两处相同的错误:写错了变量名,导致
      编译时被警告有未使用的变量。
  \item 整理本软件包所需的 \texttt{fitsio} 库和对应的头文件,与软件
      一同打包,并更新相关头文件和 \texttt{Makefile},支持 32 位与 64 位系统。
  \item 修改了 \texttt{Makefile},添加所需要的 \texttt{-fPIC} 选项,
      修改安装路径到 \texttt{/usr/local},以及其他一些小改动。
  \item 更新了说明文档。
\end{compactitem}

\noindent 2007/06/07, v1.0
\begin{compactitem}
  \item 原始程序
\end{compactitem}


\end{document}

